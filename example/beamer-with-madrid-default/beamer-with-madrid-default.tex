
% Creating a simple Title Page in Beamer
\documentclass{beamer}

\newcommand{\dhnamlibpath}{dhnamlib/}
\usepackage{\dhnamlibpath beamer-template/madrid-default}

% Section: [Your First LaTeX Presentation–Title Page]
% https://latex-beamer.com/tutorials/title-page/

% * Title page details
%   - you can use or doesn't use the optional text for the footer
%     e.g \title[title text in the footer]{This is the title for your \LaTeX{} presentation}
%     e.g \title{This is the title for your \LaTeX{} presentation}
\title[title text in the footer]{This is the title for your \LaTeX{} presentation}
\subtitle{This is the subtitle}

% * Multiple authors
%   - \and is required between authors' names
%   - "~" is used to keep first and last names together

% % authors only
% \author{First~Author \and 
%   Second~Author \and
%   Third~Author \and
%   Fourth~Author \and
%   Fifth~Author}

% % an author with an institution
% \author{latex-beamer.com}
% \institute{Online Beamer Tutorials}

% authors and institutions
\author[author text in the footer]{First~Author\inst{1} \and Second~Author~\inst{2} \and Third~Author~\inst{1}}
\institute[institution text in the footer]{
  \inst{1} Affiliation of the 1st author \and
  \inst{2} Affiliation of the 2nd author}

\date[date text in the footer]{\today}
% \date{June 29, 2022}

% Section: [Add and Position a Logo in Beamer]
% https://latex-beamer.com/tutorials/logo-beamer/

% % logo for every page
% \logo{\includegraphics[width=2cm]{image/latex-logo.pdf}}

% % logo only on title page
% \titlegraphic{\includegraphics[width=2cm]{image/latex-logo.pdf}}

% % multiple logos
% \titlegraphic{
%   \includegraphics[width=2cm]{image/latex-logo.pdf}
%   \hspace{2cm}
%   \includegraphics[width=2cm]{image/latex-logo.pdf}
%   \hspace{2cm}
%   \includegraphics[width=2cm]{image/latex-logo.pdf}
% }

% logo in a specific position
% https://latex-beamer.com/tutorials/logo-beamer/#positioning
\usepackage{tikz}
% u can use "logo" instead of "titlegraphic" to display logo for every page
\titlegraphic{ 
  \begin{tikzpicture}[overlay,remember picture]
    % "overlay,remember picture" creates an overlay above the current slide
    \node[right=0.1cm] at (current page.150){
      % logo at the left upper corner
      % the logo is located 0.1cm right from the coordinate (current page.150)
      \includegraphics[width=2cm]{image/latex-logo.pdf}};
    \node[left=0.1cm] at (current page.30){
      % logo at the right upper corner
      \includegraphics[width=2cm]{image/latex-logo.pdf}};
  \end{tikzpicture}
}

\begin{document}

% Title page frame
\begin{frame}
    \titlepage
\end{frame}

% Section: [Create Table of Contents in Beamer]
% https://latex-beamer.com/tutorials/table-of-contents/

% Outline frame
\begin{frame}{Outline or contents}
  \tableofcontents
  % \tableofcontents[hideallsubsections]  % 'hideallsubsections' hide all subsections
  % \tableofcontents[pausesections] % show tableofcontents incrementally
\end{frame}

% Highlighting each section before it begins
\AtBeginSection[ ]{
  \begin{frame}{Outline}
    \tableofcontents[currentsection]
  \end{frame}
}

\section{Problem statement}
\section{Existing results}
\subsection{Method 1}
\subsection{Method 2}
\subsection*{Method 3}  % \tableofcontents ignores \subsection*
\section{Comparative study}
\section*{References}  % \tableofcontents ignores \section*

\begin{frame}
% at least 1 frame is required
% to enforce entries in the table of contents
\end{frame}

% Section: [8 Beamer Environments you Should be Familiar With!]
% https://latex-beamer.com/tutorials/environments/

% Section: [Lists in Beamer – Complete Guide]
% https://latex-beamer.com/tutorials/lists/

% Frame example
\begin{frame}
  % [options]
  {This is the frame title}
  {This is the frame subtitle}

  This is a content

  % list example
  \begin{enumerate}
  \item One
    \begin{itemize}
    \item Sub-category
      \vspace{-0.1cm}  % space between items
    \item Sub-category
      \vspace{0.5cm}  % space between items
    \item Sub-category
    \end{itemize}
  \item Two
  \item Three
  \end{enumerate}
\end{frame}

% % % Item spacing globally
% % \setbeamertemplate{itemize/enumerate subbody begin}{\vspace{0.5cm}}
% % \setbeamertemplate{itemize/enumerate subbody end}{\vspace{1cm}}

% \begin{frame}
%   \begin{itemize}
%   \item Item one
%     \begin{itemize}
%     \item Sub item
%     \end{itemize}
%   \item Item two
%   \end{itemize}
% \end{frame}

% numbered items over multiple frames
\newcounter{currentenumi}
\begin{frame}{Lists in multiple frames}{Frame 1}
  \begin{enumerate}
  \item Item 1
  \item Item 2
  \item Item 3
    % Store the actual item number
    \setcounter{currentenumi}{\theenumi}
  \end{enumerate}
\end{frame}
\begin{frame}{Lists in multiple frames}{Frame 2}
  \begin{enumerate}
    % Use the previous stored item number
    \setcounter{enumi}{\thecurrentenumi}
  \item Item 4
  \item Item 5
  \end{enumerate}
\end{frame}


\newcommand{\titlepair}[2]{{\\ [-4ex] {\scriptsize #1} \\ [-0.0ex] #2}}

\begin{frame}{\titlepair{Super Title}{Title}}
    Some content
\end{frame}


\begin{frame}
  \begin{itemize}
  \item Item 1
    \setbeamertemplate{itemize items}[default]
  \item Item 2
    \setbeamertemplate{itemize items}[circle]
  \item Item 3
    \setbeamertemplate{itemize items}[square]
  \item Item 4.1
  \item Item 4.2
    \setbeamertemplate{itemize items}[ball]
  \item Item 5
  \end{itemize}
\end{frame}

% % Custom bullets for items
% \usepackage{pifont}
% % https://latex-beamer.com/tutorials/lists/#marker
% \begin{frame}{Pifont symbols for Beamer lists}
%   \begin{itemize}
%   \item[\ding{51}] Code 51
%   \item[\ding{56}] Code 56
%   \item[\ding{43}] Code 43
%   \item[\ding{118}] Code 118
%   \item[\ding{170}] Code 170
%   \end{itemize}
% \end{frame}

% \begin{frame}{Enumerate}
% % Alphabet, Roman and Arabic style
% % \usepackage{enumitem}
% \begin{enumerate}[label={\alph*)}]
%     \item Alphabet one
%     \item Alphabet two
% \end{enumerate}
% \begin{enumerate}[label={\roman*.}]
%     \item Roman number one
%     \item Roman number two
% \end{enumerate}
% \begin{enumerate}[label={(\arabic*)}]
%     \item Arabic number one
%     \item Arabic number two
% \end{enumerate}
% \end{frame}

% Section: [Create and Customize Columns in Beamer]
% https://latex-beamer.com/tutorials/columns/

\begin{frame}{Columns in beamer}
  \begin{columns}
    \column{0.6\textwidth}
    \centering
    This is column one with 0.75 text width.
    \column{0.4\textwidth}
    \centering
    This is column two with 0.25 text width.
  \end{columns}
\end{frame}

\begin{frame}{Text and Image in beamer}
  \begin{columns}
    \column{0.4\textwidth}
    This is an example of text and image in the same slide using columns environment.
    \column{0.6\textwidth}
    \begin{figure}
      \centering
      \includegraphics[width=\textwidth]{image/latex-logo.pdf}
      \caption{Here is the caption for the above image. }
    \end{figure}
  \end{columns}
\end{frame}

\begin{frame}{Vertical line between columns}
  \begin{columns}
    % Column 1
    \begin{column}{0.49\textwidth}
      \begin{itemize}
      \item Input layer: 2 neurons.
      \item Hidden layer: 5 neurons.
      \item Output layer: 2 neurons.
      \end{itemize}
    \end{column}

    % Column 2 (vertical line)
    \begin{column}{.02\textwidth}
      \rule{.1mm}{0.7\textheight}
    \end{column}

    % Column 3    
    \begin{column}{0.49\textwidth}
      \includegraphics[width=\textwidth]{image/latex-logo.pdf}
    \end{column}
  \end{columns}
\end{frame}

\begin{frame}{Vertical alignment}
  \begin{columns}[T]  % T: top / c: center / b: bottom
    % Column 1
    \begin{column}{0.5\textwidth}
      This is a neural network with two inputs and two outputs. It has the following parameters:
      \begin{itemize}
      \item Input layer: 2 neurons.
      \item Hidden layer: 5 neurons.
      \item Output layer: 2 neurons.
      \end{itemize}
      The neural network is drawn in \LaTeX{} using Ti\textit{k}Z package. Check latexdraw.com for more details.
    \end{column}

    % Column 2    
    \begin{column}{0.5\textwidth}
      \includegraphics[width=\textwidth]{image/latex-logo.pdf}
    \end{column}
  \end{columns}
\end{frame}

% Section: [Your Guide to Beamer Blocks]
% https://latex-beamer.com/tutorials/blocks/

% Frame 1
\begin{frame}{Basic Blocks}
  \begin{block}{Standard Block}
    This is a standard block.
  \end{block}
  \begin{alertblock}{Alert Message}
    This block presents alert message.
  \end{alertblock}
  \begin{exampleblock}{An example of typesetting tool}
    Example: MS Word, \LaTeX{}
  \end{exampleblock}
\end{frame}
% Frame 2
\begin{frame}{Mathematical Environment Blocks}
  \begin{definition} 
    This is a definition.
  \end{definition}
  
  \begin{theorem} 
    This is a theorem. 
  \end{theorem}
  
  \begin{lemma} 
    This is a proof idea.
  \end{lemma}
\end{frame}
% Frame 3
\begin{frame}{Mathematical Environment Blocks-Continued}
  \begin{proof} 
    This is a proof. 
  \end{proof}
  
  \begin{corollary}
    This is a corollary
  \end{corollary}
  
  \begin{example}
    This is an example 
  \end{example}
\end{frame}

\begin{frame}[fragile]{Mathematical Environment}
  % fragile is necessary to use \verb (verbatim) style inside a frame
  \begin{Lemma} 
    This is equivalent to \verb|lemma| block
  \end{Lemma}  
\end{frame}

% % Block stytles
% % [Default style]
% \setbeamertemplate{blocks}[default]
% % [Shadow mode of blocks]
% \setbeamertemplate{blocks}[rounded][shadow=true]

% % Change example block colors
% % 1- Block title (background and text)
% \setbeamercolor{block title example}{fg=white, bg=teal}
% % 2- Block body (background and text)
% \setbeamercolor{block body example}{ bg=teal!25}
% 
% % Change alert block colors
% % 1- Block title (background and text)
% \setbeamercolor{block title alerted}{fg=white, bg=orange}
% % 2- Block body (background and text)
% \setbeamercolor{block body alerted}{ bg=orange!25}
% 
% % Change standard block colors
% % 1- Block title (background and text)
% \setbeamercolor{block title}{bg=cyan, fg=white}
% % 2- Block body (background)
% \setbeamercolor{block body}{bg=cyan!10}

% Section: [Beamer Themes — Full List]
% https://latex-beamer.com/tutorials/beamer-themes/

% Section: [Your Beamer Guide to Text Formatting]
% https://latex-beamer.com/tutorials/text-formatting/

% \usepackage[T1]{fontenc}  % it enables the combination of bold and italics fonts
\begin{frame}{Bold, Italics and Underlining}
  This is how \textbf{bold}, \textit{italized} and
  \underline{underlined} text looks.
  You can also combine them, like \textbf{\textit{bold
      italized}}, \underline{\textbf{bold underlined}} and
  \textit{\underline{italized underlined}}.
  Finally, you can put
  \textbf{\textit{\underline{everything together}}}. % this requires \usepackage[T1]{fontenc}
\end{frame}

% % select the KP Sans-Serif font
% \usepackage[sfmath]{kpfonts}
% \begin{frame}{Slanted and small caps text}
%   This is \textsc{small caps text} and this is
%   \textsl{slanted text}.\\~\\ 
%   You can combine them, to produce \textsl{\textsc{small
%       caps slanted text}} but also \textsc{\textbf{bold small caps}} or \textsl{\underline{underlined slanted text}}.
% \end{frame}

\begin{frame}{Emphasized text}
  \emph{This} is emphasized and \textit{\emph{this} is
    also emphasized, although in a different way.}
\end{frame}

% \usepackage{bm}
\begin{frame}{Bold math example}
  Let $\bm{u}$, $\bm{v}$ be vectors and $\bm{A}$ be a
  matrix such that $\bm{Au}=\bm{v}$.
  This is a bold integral:
  \[
    \bm{\int_{-\infty}^{\infty} e^{-x^2}\,dx=\sqrt{\pi} }
  \]
\end{frame}

% \usepackage{ulem}
% https://latex-beamer.com/tutorials/text-formatting/#text-decoration
\begin{frame}{Text decorations provided by the
    \texttt{ulem} package}
  \uline{Underlined that breaks at the end of lines if
    they are too too long because the author won’t stop
    writing.} \\~\\
  \uuline{Double-Underlined text} \\~\\
  \uwave{Wavy-Underlined text} \\~\\
  \sout{Strikethrough text} \\~\\
  \xout{Struck with Hatching text} \\~\\
  \dashuline{Dashed Underline text} \\~\\ 
  \dotuline{Dotted Underline text} 
\end{frame}

% Change math font
% https://latex-beamer.com/faq/change-math-font-style/
% \usefonttheme[onlymath]{serif}
\begin{frame}{Change Math font style}
  \begin{block}{Pythagoras' theorem}
    If you square the two shorter sides in a right-angled triangle and add them together, you get the same as when you square the longest side (the hypotenuse):
    \[ x^2 + y^2 = z^2 \]
    where $x$ and $y$ are the two shorter sides and $z$ is the hypotenuse.
  \end{block}
  \textbf{Equations in Beamer: }
  \begin{align*}
    f(x) &= x^2\\
    g(x) &= \frac{1}{x}\\
    F(x) &= \int^a_b \frac{1}{3}x^3
  \end{align*}
\end{frame}

% How to change the font size on a given frame
% https://latex-beamer.com/faq/font-size-frame/
% \usepackage{lipsum}
\begin{frame}{Frame with different font sizes and spacing }{size: 9pt, vskip=10pt}
  \fontsize{9pt}{10pt}\selectfont
  \lipsum[2]
\end{frame}

% Change text color
% https://latex-beamer.com/tutorials/text-formatting/#text-color
\begin{frame}{Colors in beamer}
  {\color{blue} This is a blue block.}
  {\color{red} And this is a red one.}
  \begin{itemize}
    \color{green}
  \item This is a list.
  \item And it is colored!
  \end{itemize}

  Here I want to \colorbox{yellow}{highlight some important text in yellow}
  while leaving the rest untouched.
\end{frame}

% % Text alignment
% % https://latex-beamer.com/tutorials/text-formatting/#text-alignment
%
% % provides the \justifying command
% % \usepackage{ragged2e}
% %
% % Default alignment
% \begin{frame}{Default beamer alignment}
%   \lipsum[1]
% \end{frame}
% % Flushed right alignment
% \begin{frame}{Flushed right}
%   \begin{flushright}
%     \lipsum[2]
%   \end{flushright}
% \end{frame}
% % Centered alignment
% \begin{frame}{Centered}
%   \begin{center}
%     \lipsum[3]
%   \end{center}
% \end{frame}
% % Fully justified alignment
% \begin{frame}{Fully justified}
%   \justifying
%   \lipsum[4]
% \end{frame}

% Section: [Creating Overlays in Beamer]
% [Overlay 1]
% https://latex-beamer.com/tutorials/overlays/
\begin{frame}{Creating Overlays in Beamer}{Pause command}
  \begin{itemize}
  \item Shown from the first slide on.
    \pause
  \item Shown from the second slide on.
    \pause
  \item Shown from the third slide on.
  \end{itemize}
\end{frame}

\begin{frame}{Overlay specifications}
  \textit<1,6>{
    This will be italized text on the first slide and the sixth slide.}
  \textbf<2-4>{
    This will be bold text from the second slide to the fourth.}
  \texttt<5->{
    This will be typewriter text from the fifth slide on.}
\end{frame}

% [Overlay 2]
% https://latex-beamer.com/tutorials/overlays/2/
\begin{frame}{Onslide command}
  This is shown on all slides.
  \onslide <1-3>
  \begin{itemize}
  \item This list will only be shown
  \item on slides 1, 2 and 3.
  \end{itemize}
  This text is also be shown on slides 1, 2 and 3.

  \onslide<2>{This appears only on slide 2.}

  This appears only on slide 1 to 3.

  \onslide<1,4>
  I appear just on slides 1 and 4.

  \onslide <4-5>
    \begin{itemize}
    \item This list will only be shown
    \item on slides 4 and 5.
    \end{itemize}
\end{frame}

\begin{frame}{Only command}
  Text 1.
  \only<2>{\color{red}}  This text is red only on slide 2.
  Text 2.
  \only<4>{{\color{blue}This blue text only appears on slide 4.}}
  Text 3.
\end{frame}

\begin{frame}{Uncover command}
  First sentence.
  \uncover<2>{Second sentence.}
  \uncover<3>{Third sentence.}
  \uncover<4>{Fourth sentence.}
\end{frame}

\begin{frame}{Alternative command}
  \alt<3>{This is only on slide 3!}{This is on all other slides except slide 3.}
\end{frame}

\begin{frame}{Temporal command}
  \temporal<2>{before}{default}{after}
  \onslide<3>{}
\end{frame}

% [Overlay 3]
% https://latex-beamer.com/tutorials/overlays/3/

% Overlay specifications for mathematical environments
\begin{frame}{Overlay specifications Math environment}
  \begin{definition}<1->
    When something is repeated more than three times, we say it is spam
  \end{definition}
  \begin{example}<2->
    An example of spam is: eggs, eggs, eggs
  \end{example}
  \begin{theorem}<3->
    I don’t like spam!
  \end{theorem}
  \begin{proof}<5->
    This proof is shown after the corollary.
  \end{proof}
  \begin{corollary}<4->
    Spam is considered something bad
  \end{corollary}
\end{frame}

% Overlay commands: \only, \alt, \visible, \uncover, \invisible
% Overlay environments: onlyenv, altenv, visibleenv, uncoverenv, invisibleenv

\begin{frame}{Overlayarea environment}
  \begin{overlayarea}
    {
      \linewidth  % area width
    }
    {
      10pt % area height
    }
    \only<1>{This text is shown on the first slide.}
    \only<2>{And replaced by this one on the second.}
    \only<3->{And finally set to this text for the rest of the slides.}
    % Your content here
  \end{overlayarea}
\end{frame}

% \begin{frame}{Changing images dynamically}
%   \begin{frame}{Show three images sequentially}
%     \includegraphics<1>{1.png}
%     \includegraphics<2>{2.png}
%     \includegraphics<3>{3.png}
%   \end{frame}

%   % \begin{frame}{Show three images sequentially}
%   %   \includegraphics<1->{1.png}
%   %   \includegraphics<2->{2.png}
%   %   \includegraphics<3->{3.png}
%   % \end{frame}
% \end{frame}

% Section: [Insert a GIF into LaTeX Beamer — Short guide]
% https://latex-beamer.com/tutorials/gif-latex-beamer/

% Section: [Beamer Code Listing — Syntax highlighter]
% https://latex-beamer.com/tutorials/beamer-code/

% Section: [Beamer Font: Change its Size, Family and style]
% https://latex-beamer.com/tutorials/beamer-font/

% Section: [Beamer Table – Full guide with examples]
% https://latex-beamer.com/tutorials/beamer-table/

% Section: [Figures in Beamer – A detailed tutorial]
% https://latex-beamer.com/tutorials/beamer-figure/
 
% Section: [How to create beautiful title slide in Beamer?]
% https://latex-beamer.com/tutorials/beautiful-title-slide/

\end{document}
